
\documentclass[11pt]{article}
\usepackage{graphicx}
\begin{document}





Determine the junction resistance/Voltage bias ratio I find that the resistance of the junctions can increase a few Ωs over their room temperature values when cooled to dilution temperatures. As such, if you’re really gungho about having a 50Ω junction, you’ll do best to start with something like 48 or 49Ω. Since the goal for these
devices is an accurate assessment of the noise temperature, getting closeto 50 Ω is important, but how close is close? If your source impedance (junction impedance) is 41 Ω, then the fraction of noise power that can propagate out is 
$$
\eta= 1−|\Gamma|^2 = 0.99 
$$
so you’ll only end up with a 1% overestimation in your final system noise temperature. With the current design, your main worry will be the loss and reflections introduced by all of the intervening connectors and components. As a rule of thumb we can assume about 0.25 dB for every stage of circulation, so each double circulator will introduce 0.5 dB. All of the couplers and cables will nickel and dime you such that it is very typical toget 2 or 3 dB attenuation by the time your signal reaches your cryogenic HEMT. The bottom line is that you should worry about your wiring and component losses before getting finicky about the junction impedances. That being said, we have done our best to make our shot noise junctions very close to 50Ω anyways. In any case, our conventional wiring is intended to measure this resistance as well as is reasonable, mostly to determine the actual bias voltage that is placed across the junction. For this we like to employ a 3-probe configuration (see Figure 0.1) where we bias through a cold resistor and measure the voltage on the junction through a voltage tap. The actual resistance is unimportant in the sense given above, but the junction voltage is important for scaling voltage to temperature units in the shot noise fit. In fact, if you’re confident that your junction resistance is close enough, you can ignore it and focus on the divider ratio VJ/VB so that you know how to scale your room temperature voltage bias to the voltage that’s actually on the junction. 




This is the number that you’ll use for your fit. It’s important to get this right since the noise temperatures that you will infer will scale with your measurement of the junction voltage through $k_B$.
Figure 0.1: Bias resistor configuration for determining the ratio between the applied bias VB and the junction bias VJ. If you use a current source in place of VB you can determine the resistance and convince yourself that your reflection loss will probably be good enough.


<h3>How do you set the bias resistance values? </h3>



If you’re after noise temperature and your SNTJ is cold enough to be well into the quantum regime, then you’ll be happy if you have something like 10×hf/e, that is, something that well exceeds the corners on the quantum noise flat bottom. If you’re at 6GHz, then this turns out to be something like 25µV. Since I like to have my bias sources scaled so that they swing about ±1V for the range of interest, this means a divider ratio of something like 1:40,000. If you don’t use any room temperature dividers, you can use a single bias resistor value of 2 MΩ. I usually end up using a divider at room temperature and using something like 100kΩ so that I can measure temperature at higher temperatures as a sanity check.





Although you can,in principle, demodulate the output of your amp chain using a diode rectifier, you’ll have to play games setting the filtering of the room temperature mixer output and you’ll have to think about one more thing. I think that it’s far easier to demodulate the output of the SNTJ at a given frequency by using a standard spectrum analyzer in a triggered, zero span acquisition at the frequency that you care to measure. I use the Agilent E4407B for this purpose. Make sure to set it to “sample” each bin and not pull the max/min at each point. The E4407B also lets you set the display to linear Watts. Open up the resolution bandwidth as wide as you need to, usually less than 10 MHz on a typical instrument. This will yield the noise curve integrated over this bandwidth, so if you’re interested in getting the noise over a narrower bandwidth then you can crank it down, but be prepared to wait longer. Anyhow, with the trace averaging in the instrument you can accumulate averaged shot noise mustache traces quickly and easily by syncing a triangle sweep on the bias. Output the data however you like and perform your fit using the scaling ratio VJ/VB that you found with your 3 probe measurement above. Since these automated spectrum analyzers do a lot under the hood, you have to make sure that however it configures its own internal attenuators to level its incoming signals, that its noise floor is dominated by your amplification chain. Set up your bias sweep so that you can see a good noise mustache, sweeping out something like 10 $k_BT$  and setting the power scale on your analyzer so that the max noise fills out the y-scale. If you shut off your cryogenic HEMT, then the noise power shown on the analyzer should be very close to zero. This is an indication that the cryogenic HEMT is dominating your system noise. If it doesn’t reach zero, then you have other stuff later in the chain that is contributing significant noise and you have some troubleshooting to do.


\section{
Cold Measurement of Quantum Noise}


\section{
Plotting and Understanding Data by Eye}


\section{
Data Fit and Analysis}


\section{
Drawing Conclusions, Using what you Learned}



\section{
Sharing Data with Amplifier Noise Community}




\end{document}
