
\documentclass[11pt]{article}
\usepackage{graphicx}
\begin{document}

\section{Motivation}



    The purpose of this project is to improve the signal to noise ratio for all RF measurements in the quantum information community.  Because the scaleability of quantum computing is dependent on qubit fidelity, which in turn is dependent on RF noise level, this is a critical parameter in all superconducting qubits.  To accomplish this task, much time and money has been poured into technologies like superconducting amplifiers, circulators and other measurement tools. 
 



     While such technologies have indeed proven useful, investigation into the actual noise floor of quantum measurement experiments with shot noise tunnel junctions has shown that noise levels are not as they seem.  Losses, reflections, and subtle combinations of the two often conspire to make noise levels considerably higher than their nominal value based on the stated noise level of both the exotic quantum amplifiers now being deployed and the various commercial semiconducting amplifiers that are used to follow them. 




    This presents a problem for anyone hoping to move the overall field forward.  Money spent on both equipment and research and development for new measurement tools can get squandered when distributed across the whole QC community, based on lossy coaxial cables, multiple reflections from small impedance mismatches, and so on.  The purpose of shot noise tunnel junction metrology is thus not just to prove that it works and test some amplifiers.  It is to create a pattern of information flow across the whole field which enables all researchers to get a better handle on the losses and reflections in their measurements and then eliminiate those issues to bring their noise levels down to where they can be based on the technology available.  



The information flow of this system can be summarized as follows.  Joe Aumentado at NIST in Boulder has perfectedt the fabrication processes and a system for packaging the tunnel junctions with a good enclosure and circuit board, and the first step in using this system is acquiring a device from Joe.  Once a lab has a packaged device they need to integrate it into their measurement system in a way that gives an accurate picture of their whole measurement chain.  The core elements of this are a bias tee to allow for DC current bias of the junction while also measuring the voltage on the junction.  One thus generally wants to split the line on the DC side of the bias tee off into a voltage line and a current bias line.  Ideally both of these lines are filtered with a cryogenic DC filter of some kind such as a powder filter, and cold bias resistors should also be in both lines.



<h4>Outline of this Document</h4>
\begin{enumerate}

    \item
Introduction
    \item
Information from Joe's lab about device fab and installation
    \item
Step by Step measurement and fit guide
    \item
Notes on Units for Noise Metrology near the Quantum limit
    \item
Quantum Noise Measurements as Social Media: how creating a network of information flow in real time between quantum labs about noise measurement and improve the overall measurement standards of the whole field
\end{enumerate}


<h3>Junction design </h3>



The current NIST design (late 2012/early 2013) is the result of some trial and error over the past 6-7years. Unlike the initial work from Lafe Spietz’s thesis at Yale, this design is intended to operate at frequencies up to 10 GHz. As such, we have implemented a very broadband launch/feed in a simple CPW-G configuration. In previous versions,we had some trouble with the excess attenuation in the line on chip which led to the junction so we have minimized the square count as well as buried the aluminum junction bilayer with 500nm of copper. The copper is not intended to drive the aluminum normal at low temperatures, but is mainly intended to minimize parasitic attenuation which fouls the noise temperature measurement. In addition, we have chosen a pretty transparent oxide so that we could use smaller junction overlaps. This is necessary to move the RC rolloff of the junction > 10GHz ,i.e. 2pF for a 50 $\Omega$ junction.




    
You can verify the rolloff yourself on a network analyzer and fit it to get the capacitance. (You should check that the return loss is good at the frequency you will be measuring at anyway!) Since these are very transparent junctions, they can die pretty easily, so take the next section seriously.





<h3>Handling junctions</h3>




    Your SNTJ source probably arrived as a chip, bonded to a SMA launch sitting inside an overdesigned single-port sample box with a shorting cap on the SMA. The shorting cap is so that you don’t “blow up” the junction. Usually this means, putting enough voltage across the junction(a couple of volts) such that you punch through the nice oxide that we, at NIST, worked hard at making just right. Basically, use conventional ESD precautions when you handle these devices. Since the junctions are bonded to the box ground (go ahead, look in side), it’s hard to blow them up by touching the SMA center pin by yourself since you are probably building an equipotential between your hands the moment you grab the box. However, this doesn’t mean that whatever else you stick in that hole will be at the same potential. It’s hard to go into a full on discussion of ESD handling precautions, but if you have blown up more than a couple of junctions, sit down and do some self-examination. You should make it a habit to ground whatever touches the SMA center pin and check that when you connect voltage/current biases that  you know what potential that source is at prior to connecting. For this reason, it is often a GREAT idea to place enough dividers on a bias line such that it’s almost impossible to get the voltage too high. Resistive dividers also have the advantage in that they provide a path to ground for the junction itself.
    
<h3>Chip design and bonding</h3>




If you happen to blow up a junction, there are 5 others on the chip that you can use. The box and chip are designed such that you can slide the chip over and bond up another junction. First you’ll have to pull up the bonds on the chip that are already there. You’ll have to do this under a microscope and have steady hands, but basically, you should be able to wiggle any bond wires that stuck to the chip with some tweezers, work hardening them until they break off. Then you can blow them away. After you’ve cleaned every thing up, you can slide the chip over to the next junction.  They’re currently stuck with Apiezon N-grease, so it should be pretty easy, but be gentle. I have been putting 4 bonds on either side of the signal line for the ground connections. This is pretty tight and probably overkill. It is important however to tack down the grounds near the unused SNTJs as well as a few bonds on the opposite side, bonding the chip ground to the box ground. It might not seem like it’s necessary, I know, but you can convince your self by measuring the return loss at room temperature on a network analyzer. The seemingly superfluous bonds can make a difference in eliminating spurious resonances up above 7 GHz. In any case, you should evaluate your bonding job on the network analyzer before placing in the fridge so that you don’t have any surprises when you’re cold. If you have done everything well, you should be near -20dB return loss up to 10 GHz. If you have resonances in S11 you will end up limiting the noise power that can leave the junction resulting in noise temperature that may look excessively high at those frequencies, since the shot noise measurement can only infer the system noise temperature reckoned for the entire chain between the junction and your noise measurement instrumentation.

<h3>Box design </h3>



The box design itself is certainly overkill and is the result of some OCD on my part. It is intended to hold the magnet (which really drives the aluminum normal) and also provide a light-tight seal at the top so that the device itself is really seeing cold metal and not anything else. It is made from OFHC. It used to be gold-plated to make people feel good, but I think this is largely unnecessary. However, you will probably need to hit it with ScotchBrite a little before you bolt it on to your cold plate. It is connectorized with a single Southwest Microwave field-replaceable male SMA flanged connector(P/N213-500SF) with a 12 mil pin + teflon sleeve to feed through the 0.125” box wall. You can use any other flanged connector you like as long as it takes the 12 mil pin standard. The solid model and drawings are available to anybody that wants them. Note: we have started putting male connectors on these boxes so you can mount them directly on the RF+DC end of an Anritsu K250 bias tee without an adapter (we also have heard anecdotally that the Anritsu K251 (the one with the gold-colored body)breaks when cold!).


<h3>Embedded magnet</h3> 



    The box is as big as it is so that it can accommodate a beefy cylindrical rare earth magnet (0.375”×0.2”). Depending on what we had available this will either be one magnet or two thinner magnets stacked. As delivered, this magnet will have a little bit of Apiezon N grease gobbed into the well which will, presumably, keep it from rattling around when mounted on your fridge. If you are skeeved out by having giant magnets near your sample(are your isolators and circulators actually well-shielded, too?) then you can take out the magnet. When you’re cold you’ll have to bias well above the gap to get the noise temperature and gain and you will give up the possibility of extracting temperature. Also note that the Southwest SMA connector that we’ve mounted is slightly magnetic itself, so if you’re worried about stuff like that then you’ll have to replace it with a nonmagnetic version (like the ones from EZ-Form). If you are really paranoid you can replace the 0-80 screws holding the board down, too, but in many of the boxes these are now titanium so they’re already ok. NOTE: If you decide to not use the magnet and have to bias well above the gap, keep in mind that the resulting noise output can be enough to drive some amplifiers into bad nonlinear regimes. Specifically, Josephson parametric amplifiers with critical currents of less than 20 µA are known to saturate just above the point that the shot noise overtakes the quantum noise on the curve. If none of this makes sense, but you think you might have a related problem, send me(jose.aumentado@nist.gov)a note!

<h3>Noise power formula</h3>



The formula for the noise power emitted by the shot noise tunnel junction (SNTJ) in the quantum regime (kBT < hf) is: 
$$
N= Gk_BBT_N + \left(\frac{eV + hf}{2k_BT}\right)
\coth{\left(\frac{eV + hf}{2k_BT}\right)} + \left(\frac{eV - hf}{2k_BT}\right) \\ 
\coth{\left(\frac{eV - hf}{2k_BT}\right)}
$$

G is the power gain of the entire system, kB is the Boltzmann constant, B is the bandwidth, TN is  the noise temperature, V is the junction voltage, h is Planck’s constant and f is the frequency at which the noise is measured. This is the full “double-coth” functi on which yields the quantum noise flat-bottom feature. Although the double-coth is overkill for higher temperatures, you can still use whatever fit routine you cookedup for the double-coth fit, it just might take a little longer. In the end,however, if you are just concerned with noise temperature, then the quantum noise stuff and even the temperature don’t matter. You can fix the temperature to something close and just do linear fits on the asymptotes to extract the gain and the noise temperature.


<h3>Operation</h3>



Determine the junction resistance/Voltage bias ratio I find that the resistance of the junctions can increase a few Ωs over their room temperature values when cooled to dilution temperatures. As such, if you’re really gungho about having a 50Ω junction, you’ll do best to start with something like 48 or 49Ω. Since the goal for these
devices is an accurate assessment of the noise temperature, getting closeto 50 Ω is important, but how close is close? If your source impedance (junction impedance) is 41 Ω, then the fraction of noise power that can propagate out is 
$$
\eta= 1−|\Gamma|^2 = 0.99 
$$
so you’ll only end up with a 1% overestimation in your final system noise temperature. With the current design, your main worry will be the loss and reflections introduced by all of the intervening connectors and components. As a rule of thumb we can assume about 0.25 dB for every stage of circulation, so each double circulator will introduce 0.5 dB. All of the couplers and cables will nickel and dime you such that it is very typical toget 2 or 3 dB attenuation by the time your signal reaches your cryogenic HEMT. The bottom line is that you should worry about your wiring and component losses before getting finicky about the junction impedances. That being said, we have done our best to make our shot noise junctions very close to 50Ω anyways. In any case, our conventional wiring is intended to measure this resistance as well as is reasonable, mostly to determine the actual bias voltage that is placed across the junction. For this we like to employ a 3-probe configuration (see Figure 0.1) where we bias through a cold resistor and measure the voltage on the junction through a voltage tap. The actual resistance is unimportant in the sense given above, but the junction voltage is important for scaling voltage to temperature units in the shot noise fit. In fact, if you’re confident that your junction resistance is close enough, you can ignore it and focus on the divider ratio VJ/VB so that you know how to scale your room temperature voltage bias to the voltage that’s actually on the junction. 




This is the number that you’ll use for your fit. It’s important to get this right since the noise temperatures that you will infer will scale with your measurement of the junction voltage through $k_B$.
Figure 0.1: Bias resistor configuration for determining the ratio between the applied bias VB and the junction bias VJ. If you use a current source in place of VB you can determine the resistance and convince yourself that your reflection loss will probably be good enough.


<h3>How do you set the bias resistance values? </h3>



If you’re after noise temperature and your SNTJ is cold enough to be well into the quantum regime, then you’ll be happy if you have something like 10×hf/e, that is, something that well exceeds the corners on the quantum noise flat bottom. If you’re at 6GHz, then this turns out to be something like 25µV. Since I like to have my bias sources scaled so that they swing about ±1V for the range of interest, this means a divider ratio of something like 1:40,000. If you don’t use any room temperature dividers, you can use a single bias resistor value of 2 MΩ. I usually end up using a divider at room temperature and using something like 100kΩ so that I can measure temperature at higher temperatures as a sanity check.


\section{
Measure the noise }



Although you can,in principle, demodulate the output of your amp chain using a diode rectifier, you’ll have to play games setting the filtering of the room temperature mixer output and you’ll have to think about one more thing. I think that it’s far easier to demodulate the output of the SNTJ at a given frequency by using a standard spectrum analyzer in a triggered, zero span acquisition at the frequency that you care to measure. I use the Agilent E4407B for this purpose. Make sure to set it to “sample” each bin and not pull the max/min at each point. The E4407B also lets you set the display to linear Watts. Open up the resolution bandwidth as wide as you need to, usually less than 10 MHz on a typical instrument. This will yield the noise curve integrated over this bandwidth, so if you’re interested in getting the noise over a narrower bandwidth then you can crank it down, but be prepared to wait longer. Anyhow, with the trace averaging in the instrument you can accumulate averaged shot noise mustache traces quickly and easily by syncing a triangle sweep on the bias. Output the data however you like and perform your fit using the scaling ratio VJ/VB that you found with your 3 probe measurement above. Since these automated spectrum analyzers do a lot under the hood, you have to make sure that however it configures its own internal attenuators to level its incoming signals, that its noise floor is dominated by your amplification chain. Set up your bias sweep so that you can see a good noise mustache, sweeping out something like 10 $k_BT$  and setting the power scale on your analyzer so that the max noise fills out the y-scale. If you shut off your cryogenic HEMT, then the noise power shown on the analyzer should be very closeto zero. This is an indication that the cryogenic HEMT is dominating your system noise. If it doesn’t reach zero, then you have other stuff later in the chain that is contributing significant noise and you have some troubleshooting to do.


\section{
HEY! My system noise looks higher than the spec sheet!} 




Your first inclination might be to think that somehow the SNTJ is giving you a wrong answer, probably because the noise looks high and you really want to go out and say that you have quantum-limited added noise. Before you come to this conclusion, you have to remember what the SNTJ gives you—it can only yield the total system noise between the junction and the spectrum analyzer. This includes loss contributions inside the SNTJ box between the junction and the SMA, all of the reflections at every connection, the loss in every circulator, bias tee, and normal metal cable as well as the temperature of those cables. If you are unsure about what these pieces add, or skeptical that they should add so much, then sit down and calculate and then show someone else as a check. We are happy to help you figure out what’s going on. As a point of reference,though,be aware that many of the system noise numbers that many groups have stated in their papers are just plain wrong.  It is typical for instance, fora CalTech HEMT amplifier to have 5K noise temperature, but after all of the loss between a DUT at the mixing chamber and the input of the HEMT at 4 K, we find it typical to see between 10-15 K total system noise.

<h3>Attribution</h3> 



Some groups have asked how to attribute credit to us (co-authorship, acknowledgement, or nothing at all?). It really depends on what piece the SNTJ plays in your experiment, but in the end we should just have a conversation to make sure we’re all on the same page.


<h3>More stuff?</h3> 




These are just random notes off the top of my head. If you have other tips and suggestions that you think should be included, then feel free to write something up and send it to me at jose.aumentado@boulder.nist.gov and I’ll work on including it. Also,if you would like example Matlab code+data,just send us a note.




Determine the junction resistance/Voltage bias ratio I find that the resistance of the junctions can increase a few Ωs over their room temperature values when cooled to dilution temperatures. As such, if you’re really gungho about having a 50Ω junction, you’ll do best to start with something like 48 or 49Ω. Since the goal for these
devices is an accurate assessment of the noise temperature, getting closeto 50 Ω is important, but how close is close? If your source impedance (junction impedance) is 41 Ω, then the fraction of noise power that can propagate out is 
$$
\eta= 1−|\Gamma|^2 = 0.99 
$$
so you’ll only end up with a 1% overestimation in your final system noise temperature. With the current design, your main worry will be the loss and reflections introduced by all of the intervening connectors and components. As a rule of thumb we can assume about 0.25 dB for every stage of circulation, so each double circulator will introduce 0.5 dB. All of the couplers and cables will nickel and dime you such that it is very typical toget 2 or 3 dB attenuation by the time your signal reaches your cryogenic HEMT. The bottom line is that you should worry about your wiring and component losses before getting finicky about the junction impedances. That being said, we have done our best to make our shot noise junctions very close to 50Ω anyways. In any case, our conventional wiring is intended to measure this resistance as well as is reasonable, mostly to determine the actual bias voltage that is placed across the junction. For this we like to employ a 3-probe configuration (see Figure 0.1) where we bias through a cold resistor and measure the voltage on the junction through a voltage tap. The actual resistance is unimportant in the sense given above, but the junction voltage is important for scaling voltage to temperature units in the shot noise fit. In fact, if you’re confident that your junction resistance is close enough, you can ignore it and focus on the divider ratio VJ/VB so that you know how to scale your room temperature voltage bias to the voltage that’s actually on the junction. 




This is the number that you’ll use for your fit. It’s important to get this right since the noise temperatures that you will infer will scale with your measurement of the junction voltage through $k_B$.
Figure 0.1: Bias resistor configuration for determining the ratio between the applied bias VB and the junction bias VJ. If you use a current source in place of VB you can determine the resistance and convince yourself that your reflection loss will probably be good enough.


<h3>How do you set the bias resistance values? </h3>



If you’re after noise temperature and your SNTJ is cold enough to be well into the quantum regime, then you’ll be happy if you have something like 10×hf/e, that is, something that well exceeds the corners on the quantum noise flat bottom. If you’re at 6GHz, then this turns out to be something like 25µV. Since I like to have my bias sources scaled so that they swing about ±1V for the range of interest, this means a divider ratio of something like 1:40,000. If you don’t use any room temperature dividers, you can use a single bias resistor value of 2 MΩ. I usually end up using a divider at room temperature and using something like 100kΩ so that I can measure temperature at higher temperatures as a sanity check.





Although you can,in principle, demodulate the output of your amp chain using a diode rectifier, you’ll have to play games setting the filtering of the room temperature mixer output and you’ll have to think about one more thing. I think that it’s far easier to demodulate the output of the SNTJ at a given frequency by using a standard spectrum analyzer in a triggered, zero span acquisition at the frequency that you care to measure. I use the Agilent E4407B for this purpose. Make sure to set it to “sample” each bin and not pull the max/min at each point. The E4407B also lets you set the display to linear Watts. Open up the resolution bandwidth as wide as you need to, usually less than 10 MHz on a typical instrument. This will yield the noise curve integrated over this bandwidth, so if you’re interested in getting the noise over a narrower bandwidth then you can crank it down, but be prepared to wait longer. Anyhow, with the trace averaging in the instrument you can accumulate averaged shot noise mustache traces quickly and easily by syncing a triangle sweep on the bias. Output the data however you like and perform your fit using the scaling ratio VJ/VB that you found with your 3 probe measurement above. Since these automated spectrum analyzers do a lot under the hood, you have to make sure that however it configures its own internal attenuators to level its incoming signals, that its noise floor is dominated by your amplification chain. Set up your bias sweep so that you can see a good noise mustache, sweeping out something like 10 $k_BT$  and setting the power scale on your analyzer so that the max noise fills out the y-scale. If you shut off your cryogenic HEMT, then the noise power shown on the analyzer should be very close to zero. This is an indication that the cryogenic HEMT is dominating your system noise. If it doesn’t reach zero, then you have other stuff later in the chain that is contributing significant noise and you have some troubleshooting to do.


\section{
Cold Measurement of Quantum Noise}


\section{
Plotting and Understanding Data by Eye}


\section{
Data Fit and Analysis}


\section{
Drawing Conclusions, Using what you Learned}



\section{
Sharing Data with Amplifier Noise Community}



	Noise temperature of microwave measurements near the quantum limit is becoming an increasingly critical issue in quantum information science.  Building and deploying ultra-low-noise amplfifiers is a key component to improving overall performance of this field, however simply building better amplifiers is not enough.  In order to realize the full potential of better amplifiers, as well as to continue to push the amplifier field ahead, quality measurements of noise are needed.   Many times multiple labs will buy the exact same amplifier only to find that one lab has twice the measured noise performance as another because of details of the measurement chain.  In this application note we discuss the physics and engineering needed to accurately measure the noise of a measurement chain.


\section{
Noise Temperature}



    		For the purpose of this application note, all components are assumed to be impedance matched to a 50 $\Omega$ transmission line.  We start by discussing very briefly how noise temperature is defined for amplifiers well above the quantum noise limit at temperatures well above the quantum temperature(both of these will be defined below).  





    	In this high temperature regime it makes sense to define noise temperature the following way:  the noise temperature of an amplifier is the temperature of a matched load at the input the thermal noise of which exactly doubles the noise observed at the output.  For example, if an amplifier has 20 dB of gain and a 10 K noise temperature and a 10 K matched load were at the input one would measure a noise power at the output of 2000 K, or 10 K for the load plus 10 K for the amplifier times 100 for the power gain.  This definition is both aesthetically pleasing and simple to understand experimenatally.  As we shall see, however it is non-useful at very low noise levels and has led to decades of confusion.  





	One important point to note about this noise temperature is that it is a completely fictitious temperature.  There is no implied object in the amplifier which is actually physically at the noise temperature.  A room temperature amplifier can, for instance, have a noise temperature of just a few 10's of kelvin.  Temperature units are used because they are convenient way to describe white noise and because they are easy to think about in the context of both low temperature physics and radio astronomy, but they should not be taken too seriously as a physical property(although, confusingly, noise temperatures often scale linearly with temperature).  




    	One subject that is constantly confusing for people who are new to amplfiier noise measurements is how to deal with system noise in a multi-section measurement chain.  Whether one is dealing with communications, radio astronomy or low temperature physics, the amplifier in question is never alone in the measurement chain.  There are generally cables, directional couplers, circulators, multiple stages of amplifier, attenuators, as well as various impedance mismatches.  




    	The most important of these cases is that of multiple stages of amplifier.  When analyzing a chain of amplifiers, the key fact is that one must multiply by linear power gain going forward in the chain and divide by linear power gain going backward through the chain.  In quantitative terms, if a pair of amplifiers have gains $G_1$ and $G_2$ and noise temperature $T_{n1}$ and $T_{n2}$, the total gain is $G_1G_2$ and the total noise temperature is $T_{n1} + T_{n2}/G_1$.  Note that if the gain of the first stage amplifier is high, the noise of the second stage has very little affect on the overall noise of the combined amplifier chain.  For instance if a cryogenic amplifier has 40 dB of gain, if the following amplifier at room temperature has 10,000 K noise temperature it only adds 1 K of noise to the total measurement chain!  This shows why the first stage is so much more critical than any other part of the measurement chain.  If the first stage has high enough gain and low enough noise, the demands for noise on the rest of the chain can be relatively low.  





    	Most other impedance-matched elements which add loss can be modelled as attenuators, so we consider those next.  An attenuator with linear attenuation factor A may be considered to be an amplifier with power gain less than one (G = 1/A), with a noise temperature equal to the physical temperature times A - 1.  The noise temperature of a amplifier with an attenuator at its input is then $T^{attenuator}_n + AT^{amplifier}_{n} = T^{attenuator}_{physical}(A-1) + AT_n^{amp}$, where A is the linear power attenuation, $T^{attenuator}_{physical}$ is the physical temperature of the attenuator, and $T_n^{amplifier}$ is the amplifier noise temperature.  This is why minimizing loss at the input of an amplifier chain is so critical.  For example, a 3 dB loss at the input of an amplifier, even if it is at a negligible temperature, adds a factor of at least 2 if not more(depending on the attenuator temperature, which could add another factor of 2 or more) to the overall system noise temperature!  It is quite common for low temperature physicists using exactly the same cryogenic low noise amplifier to see as much as a factor of four difference in observed system noise temperature due to small differences in how microwave loss is handled for this reason.  




    	Although it is described at length in the literature it is worth describing here the nature of noise from a impedance-matched resistor including quantum mechanical effects, because we will rely on that extensively in this dicsussion.  The noise power $S_I\times 50 \Omega$ (in Watts per Hz, which has units of energy) from a matched load at temperature T and angular frequency $\omega$ is
	
$$P(\omega,T)=\frac{\hbar\omega}{2}\coth{\left(\frac{\hbar\omega}{2kT}\right)}.$$


Note that in the limit of high temperature and low frequency is kT.  In the limit of low temperature and high frequency $\frac{\hbar\omega}{2}$.  For the purpose of this discussion we prefer to use units of photons per mode rather than power.  To convert to photons per mode we divide energy by $\hbar\omega$, to get the "noise number", 
$$n(\omega,T)=\frac{1}{2}coth{\left(\frac{\hbar\omega}{2kT}\right)}.$$


<h4>Historical Background and Erroneous Definitions</h4>



    	Before defining the noise units we will use we must discuss some of the erroneous definitions of noise temperature which exist in the literature and cause extensive confusion.  One of the major sources of confusion is the high temperature definition used above in this note.  Defining the noise temperature in terms of the <i>temperature</i> of a load at the input of the amplifier becomes highly problematic in the case where kT is near $\hbar\omega$.  Naive application of this exact definition has led to several non-useful definitions for the quantum limit and for noise temperature.  By solving for the temperature of a quantum resistor in the above expression, some authors have used definitions of noise temperature which are not linear in the amplifier noise power, and which add factors of either $\ln{2}$ or $\ln{3}$ depending on whether the vacuum noise is ignored or not.  Furthermore, there are authors who rather than ignore the vacuum noise, count it twice, including both the added noise of the amplifier and the quantum noise of the matched load in the total stated noise, adding a factor of 2 to the classical definition above.  It is not our purpose to delve deeply into the literature here to account for all the published errors(I will do that elsewhere).   However, we wish to warn the reader that grievous errors and inconsistencies exist in parts of the amplifier literature both from physics and from engineering journals, and that great care must be taken when attempting to interpret the results of some papers.  

<h4>Proper Units for Quantum Amplifier Metrology</h4>




    	Due to the level of confusion in the literature, great care must be taken in defining the units for measuring amplifier noise.  The unit we describe here is one which we claim is used colloquially by almost the entire amplifier community but which is almost never stated explicitly.  Often the incorrect definition used above is stated in the same paper where in the body of the paper a more consistent definition is used in practice.  





    	We define the noise temperature of an amplifier by considering a amplifier with a matched load at the input such that the noise at the output is half from the load and half from added amplifier noise.  In this situation, 
$$T_N = \frac{P}{2kG},$$
	where P is the noise power per unit of bandwidth measured at the output of the amplifier, G is the linear power gain of the amplifier and k is the Boltzmann constant.   In the situations where the noise from the resistor is completely thermal, this definition is identical to the classical definition.  But it has the important difference that as we get into the quantum regime the noise temperature in kelvin units continues to be linear in added noise power.  




 
    	In addition to noise temperature and noise number we now mention a third unit which is used in the amplifier industry, the noise figure.  The noise figure is a way of comparing noise temperature to 290 K, expressed in dB.  Thus the noise figure NF is 





    $$NF = 10 \log_{10}{\left(1 + \frac{T_N}{290}\right)}.$$




    While this figure of merit is not very useful at low temperatures and is less favored by physicists than engineers, it is important to include it here for completeness.  Many amplifier companies specify their amplifiers in noise figure and being able to quickly compare this to noise temperature is useful.  To that end, we include here a short table comparing noise figure, noise temperature, and noise number at 5 and 10 GHz.  




    There are a couple of things worth noting about these numbers.  First of all note that this noise number is, as we will see below, not the factor by which the amplifier's noise is above the quantum limit.  An amplifier with a noise number of 1 is twice the quantum limit rather than exactly at the quantum limit.  Second of all, the noise number is, as with noise temperature, <i>added</i> noise from the amplifier, not the total noise referenced to the input.  If the input is a matched load in equilibrium at zero temperature, it will add at least 1/2 to the observed noise referenced to the input.  One of the errors in some papers in the literature is including this 1/2 in the quantum limit, while still mixing units with temperature units.  Again, always be careful to determine which of the various definitions the author of any given paper is using.




    	There is a Heisenberg uncertainty relation between the two phase quadratures of a measurement of a quantum signal. We will not delve into the physical details of this here, and they are very thoroughly covered in the references(ref).  We will only state the relevant limits on amplifiers from an engineering perspective.  




    	We first define the somewhat confusing standard nomenclature used in the literature.  ``Phase insensitive'' refers to amplifiers that amplify a signal without regard to the phase of the signal.  This describes almost all commercially available and high temperature amplifiers, as well as amplifiers based on the DC-SQUID and the non-degenerate parametric amplifier.  ``Phase sensitive'' amplifiers have gain which is different for the different phase quadratures.  This applies to degenerate parametric amplifiers.  This distinction is critical from an engineering perspective, because when gain depends on phase it is possible to measure one of the quadratures with arbitrary precision by throwing away precision in the other quadrature.  




    	For phase insensitive amplifiers, the uncertainty relation implies the following limit on the noise number n:
$$n\geq \frac{1}{2}\left|1 - \frac{1}{G}\right|,$$
where G is the power gain of the amplifier.  Note that if the gain is 1, as for a lossless transmission line, no noise must be added.  For large gains, this relation means that all phase insensitive amplifiers must add at least half a photon of noise per mode.  Translating this into a minimum possible noise temperature in kelvin at different frequencies is useful, and we do that in the follwing table.  





    It is worth noting that as with measuring position without momentup or energy without phase, if we only measure one phase quadrature there is no quantum limit on the precision.  This allows us to use the troubling phrase ``below the standard quantum limit'' to describe the performance of amplifiers which have a noise in one quadrature below half a photon per mode.  This can be very useful for measurement of harmonic signals where the phase of the signal is known and one does not need any information from the other quadrature.  Note that if the harmonic signal is not squeezed, that there is still always a half photon added from the quantum noise from the signal, so squeezing of this signal would be useful to get the ultimate maximum possible precision in measurement.  




    	Here we discuss the effect of amplifier noise on practical measurement time.  When measuring an effective temperature T with added noise $T_n$, the imprecision in the measurement in temperature units is given by the Dicke radiometer formula
$$\delta T = \frac{T + T_n}{\sqrt{B\tau}},$$
where B is the bandwidth of the measurement and $\tau$ is the integration time.  This formula is important since it shows the power of noise temperature in determining integration time.  If we assume that we want a fixed signal to noise ratio, and that amplifier noise dominates the overall noise, the time required to achieve that signal to noise ratio scales with the <i>square</i> of the noise temperature.  Thus even a factor of order 3 improvement in noise temperature can shorten integration times by a factor of 10 to get the same performance.  Note that this consideration also applies to measurement of harmonic signals with no added noise, since the white noise background from the amplifier still integrates down according to this formula.  





    We briefly digress to discuss how to estimate noise thermometer performance.  By simply substituting the estimated temperature and noise temperature and bandwidth into the above formula, we can get a total number for temperature uncertainty in K/$\sqrt{Hz}$, which makes estimation of measurement times and errors simple.  For instance in a typical microwave noise thermometry measurement with 10 K of added noise and 100 MHz bandwidth, the uncertainty is about 30 mK/$\sqrt{Hz}$



\section{
Practical Noise Determination}
	<h4>Y-Factor Measurement</h4>



	The Y-factor method invovles measuring the output noise of a measurement system at two known temperatures and calculating the noise and gain from those noise values.  This is generally used in the classical regime where noise depends linearly on temperature.  In the classical regime, we consider the output noise to be $P_{1,2} = G(T_n + T_{1,2})$, where G is gain in units of output power per kelvin of noise, $T_n$ is amplifier noise temperature, and $T_{1,2}$ are the two physical temperatures.  It is then straight foward to compute the noise temperature from two noise power measurements.  
$$T_n = \left(\frac{T_2 - T_1}{P_2 - P_1}\right)P_1 - T_1.$$

	This method has several problems for people doing measurements near the quantum limit.  First of all, if both temperatures are in the quantum regime there is very little difference in noise power.  At 10 GHz, a 10 mK load and a 20 mK load are almost identical in noise power, for instance, and much higher temperatures are needed to get any useful data.  In the quantum case, we can write the photon number noise at the output as $P_{1,2} = G(n + \frac{1}{2}\coth{\frac{\hbar\omega}{2kT}})$.  This yields the following awkward formula for noise number
$$n = \left(\frac{\frac{1}{2}\coth{\left(\frac{\hbar\omega}{2kT_2}\right)} - \frac{1}{2}\coth{\left(\frac{\hbar\omega}{2kT_1}\right)}}{2(P_2 - P_1)}\right)P_1 - \frac{1}{2}\coth{\left(\frac{\hbar\omega}{2kT_1}\right)}.$$



	Given that at very low temperatures the noise levels of different temperatures are almost identical, one needs to have one temperature be much higher than the other to get over the quantum temperature.  Heating to significant fractions of a kelvin or over a kelvin in a dilution refrigerator can be very disruptive and can lead to errors due to excessive heating.  Also, going from one stable temperature to another in a low temperature apparatus can be very time consuming, and when large numbers of data points are needed to characterize the noise of an amplifier, this can lead to huge time investments.  

<h4>Room Temperature Noise Measurements</h4>


	
	Another very common method of measuring noise is to use a calibrated active noise source.  There are a variety of commercially avialable sources used for this purpose.  The commercial sources which are widely available are at room temperature.  This neccesitates using both extensive attenuators and cables to get from the source to the input of the cryogenic amplifier, with the number of components increasing as the amplifier gets closer and closer to the quantum limit.  The final accuracy of the noise at the input of the amplifier is then dependent on knowing the exact attenuation and temperature of every stage of the measurement chain from the source to the amplifier.  While these kinds of systems can be very efficient for noise calibration at room temperature and at moderate cryogenic temperatures they are extremely impractical for ultra-low-noise cryogenic amplifiers near the quantum limit.  

	<h4>Shot Noise Sources</h4>
	



	The problems listed above with standard noise calibration techniques are solved by using the various shot noise sources which are available.  Shot noise sources can be very close to the input of the amplifier, can by compact and can work at any temperature.  They can allow for very rapid and accurate measurement of noise.  Some subtleties must be understood, however, if the accuracy is to be realized.  




	The primary device we will discuss here is the Shot Noise Tunnel Juntion(SNTJ).  This device is a matched 50 $\Omega$ load where the origin of the resistance is a tunnel junction.  By applying a DC bias current to the junciton and measuring the voltage, it is possible to have a very accurate knowledge of the noise based on well-understood physics.  
$$P = T_N + \frac{1}{2}T\left[   \left(\frac{V+F}{2T}\right) \coth{\left(\frac{V+F}{2T}\right)} + \left(\frac{V-F}{2T}\right) \coth{\left(\frac{V-F}{2T}\right)}    \right]$$





     	For a tunnel junction at finite frequency, the output noise number is 
$$n_{SNTJ} = \frac{1}{4}\left( u + 1\right)\coth{\left(\frac{u+1}{2t}\right)} + \frac{1}{4}\left( u - 1 \right)\coth{\left(\frac{u-1}{2t}\right)} ,
$$where u is the normalized voltage $\frac{eV}{\hbar\omega}$ and t is the normalized temperature $\frac{kT}{\hbar\omega}$. In the limit of low frequency and high temperature, this expression reduces to 
$$n_{SNTJ} = \frac{1}{2}u\coth{\left(\frac{u}{2t}\right)} = \frac{eV}{2\hbar\omega}\coth{\left(\frac{eV}{2kT}\right)}, 
$$
or
$$
T_{SNTJ} = \frac{eV}{2k}\coth{\left(\frac{eV}{2kT}\right)}
$$


By measuring the noise at the output of the measurement chain at several voltages, one may readily determine the full system noise temperature or noise number.  This can be done by either measuring two points in the fully linear range of the curve or by measuring a larger number of points and fitting them to the functional form above.  The plot of normalized noise number at the output of the amplifier shows conceptually how the added noise from the measurement manifests itself as an offset in the curve.  Since the bias point can be changed in a very small fraction of a second, it is possible for low noise amplfiers to acquire noise data as a function of various parameters very quickly.  



The equation for the noise number at the output of the whole measurement chain is
$$
n = G(n_{amp} + n_{SNTJ}),
$$
where $n_{amp}$ is the added noise number from the amplifier and $n_{SNTJ}$ is as defined above.  Looking at typical theoretical noise curve plotted here we see that all of the information about temperature is in the voltage range where $eV\approx\hbar\omega$.  Below that voltage, there is very little information about either amplifier noise or temperature.  A straightforward way to measure noise temperature or noise number is to measure the noise at two voltages, both on the linear range of the curve.  Good choices for an accurate linear measurement are(cite thesis) $(\hbar\omega + 5 kT)/e$ and $(\hbar\omega + 10 kT)/e$.  This is very much like a y-factor measurement.  

	In this regime, the noise number from the junction is 
$$
n_{SNTJ} =  \frac{1}{2}u,
$$
and the noise temperature is 
$$
T_{SNTJ} = \frac{eV}{2k},
$$
so from two points both on the linear regime, if we measure output noises $n_1$ and$n_2$, the added system noise number is 
$$
n_{system} = \frac{u_2 -  u_1}{n_2 - n_1}n_2 - \frac{1}{2},
$$
and the added system noise temperature is 
$$
T_{system} = \frac{eV_2 - eV_1}{2kT_2 - 2kT_1}T_2 - \frac{\hbar\omega}{2k}.
$$

	With a shot noise source in hand, writing data acquisition code to get these numbers quickly is relatively straigthforward.  Note that in the quantum regime it is important to use the frequency-appropriate formula for each frequency.  
	

\section{
Noise Metrology as Social Media}



    Social media has fundamentally altered the Human Condition for all people on the Planet.  Communication in politics, war, the arts, in social interactions and in business all now take place at the pace of Twitter.  And yet science continues to publish on a 18 month time scale.  We can claim this is because we are more careful and subject our work to more review and editing than others, but this is a red herring.  




    On a daily basis we are generating information which is neither proprietary nor secret, is too small and not new enough to warrant publication and yet which can benefit the overall progress of our field enormously.  The lack of real time technical communication throughout the community in a quantitative and math-friendly format which can be interacted with on smart phones is therefore holding back all scientific progress in comparison to other human endeavors today.  




    The shot noise amplifier metrology being developed at APL can benefit enormously from the use of the Web to publish information in real time.  The goal of this project is not to create one single result which will be shared in a paper, or a single technology which is to be physically distributed.  Rather, a successful system will involve a continuous flow of information and materials throughout the community of quantum information research labs.  In terms of physical objects, this flow will consist of devices shipped from the Aumentado lab at NIST, packages and boards from various outsourced suppliers, supplies of quantum amplifiers from research labs and startup companies, commercial HEMT amps, and various cryogenic microwave components.  Right now success in this field is often dependent on so-called "lore": word of mouth information about what components will either fail cold or have small problems which affect overall performance. 




    Given the tight-knit nature of the quantum information and measurement community it should be possible to replace this "lore" of rumors told in emails and verbally with real time quantitative information shared on the Web between labs.  What we propose to do here is to create the infrastructure to do just that: to share information in real time at all phases of the workflow from purchasing components to fabrication, to exchanging of samples between labs, testing with the shot noise junctions, further metrology using qubit readouts, and use of this information for further improveing the measurement.  It is a goal of this work for no lab to create an improvement in measurement noise floor alone, but to always be in contact with all other researchers in a constant exchange of information between the people actually gathering that information as it happens: grad students, post docs, technicians etc.  When a grad student is stuck on a 6 month snag with a mysteriously broken measurement they should have every single other grad student in the field potentially as backup, helping solve their problems in real time.  This can provide exponential speedup to progress, as the current waste of benchtop researcher time trying to fix broken experiments that have the same failure mode another researcher already found but didn't publish is potentially huge.  




    We note that this new class of information channel does not obviate the need for existing systems of publication: peer reviewed papers, white papers, PhD dissertations, etc.  In fact, it has the potential to create a vastly more efficient and effective peer review system, in which we are all able to see each other's work and not just evaluate it as we currently do but actually make it better, so that no one ends up "stranded" in a obscure and broken part of the field where nothing works and no one trusts their results.  



\section{
Summary of Proposed System}




    It is our intent to create a system for rapid mobile-based sharing on the Web which tracks the tunnel junction based shot noise metrology experiment all the way through the workflow of that type of measurement.  The stages of this workflow that we must deal with are:
    

    \begin{enumerate}

        \item
Junction design
        \item
Junction Fabrication
        \item
Junction distribution
        \item
Package fabrication
        \item
package distribution
        \item
board fabrication
        \item
board distribution
        \item
Installation in fridge
        \item
Overall measurement chain assembly and documentation
        \item
Tests before and during cooldown
        \item
Data at base temperature
        \item
Plotting data
        \item
Interpreting data for future improvements
    \end{enumerate}
    
    


        At each phase of this workflow the proposed system will have a method for rapid publication directly to the Web for sharing with peers.  The Geometron geometric metalanguage, developed independently by Lafelabs LLC, allows for rapid creation of graphics using a specific graphical language(an instance of the metalanguage).   
    




\end{document}
