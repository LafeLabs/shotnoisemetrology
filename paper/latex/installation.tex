
\documentclass[11pt]{article}
\usepackage{graphicx}
\begin{document}

\section{
Introduction}

<h3>Junction design </h3>



The current NIST design (late 2012/early 2013) is the result of some trial and error over the past 6-7years. Unlike the initial work from Lafe Spietz’s thesis at Yale, this design is intended to operate at frequencies up to 10 GHz. As such, we have implemented a very broadband launch/feed in a simple CPW-G configuration. In previous versions,we had some trouble with the excess attenuation in the line on chip which led to the junction so we have minimized the square count as well as buried the aluminum junction bilayer with 500nm of copper. The copper is not intended to drive the aluminum normal at low temperatures, but is mainly intended to minimize parasitic attenuation which fouls the noise temperature measurement. In addition, we have chosen a pretty transparent oxide so that we could use smaller junction overlaps. This is necessary to move the RC rolloff of the junction > 10GHz ,i.e. 2pF for a 50 $\Omega$ junction.




    
You can verify the rolloff yourself on a network analyzer and fit it to get the capacitance. (You should check that the return loss is good at the frequency you will be measuring at anyway!) Since these are very transparent junctions, they can die pretty easily, so take the next section seriously.





<h3>Handling junctions</h3>




    Your SNTJ source probably arrived as a chip, bonded to a SMA launch sitting inside an overdesigned single-port sample box with a shorting cap on the SMA. The shorting cap is so that you don’t “blow up” the junction. Usually this means, putting enough voltage across the junction(a couple of volts) such that you punch through the nice oxide that we, at NIST, worked hard at making just right. Basically, use conventional ESD precautions when you handle these devices. Since the junctions are bonded to the box ground (go ahead, look in side), it’s hard to blow them up by touching the SMA center pin by yourself since you are probably building an equipotential between your hands the moment you grab the box. However, this doesn’t mean that whatever else you stick in that hole will be at the same potential. It’s hard to go into a full on discussion of ESD handling precautions, but if you have blown up more than a couple of junctions, sit down and do some self-examination. You should make it a habit to ground whatever touches the SMA center pin and check that when you connect voltage/current biases that  you know what potential that source is at prior to connecting. For this reason, it is often a GREAT idea to place enough dividers on a bias line such that it’s almost impossible to get the voltage too high. Resistive dividers also have the advantage in that they provide a path to ground for the junction itself.
    
<h3>Chip design and bonding</h3>




If you happen to blow up a junction, there are 5 others on the chip that you can use. The box and chip are designed such that you can slide the chip over and bond up another junction. First you’ll have to pull up the bonds on the chip that are already there. You’ll have to do this under a microscope and have steady hands, but basically, you should be able to wiggle any bond wires that stuck to the chip with some tweezers, work hardening them until they break off. Then you can blow them away. After you’ve cleaned every thing up, you can slide the chip over to the next junction.  They’re currently stuck with Apiezon N-grease, so it should be pretty easy, but be gentle. I have been putting 4 bonds on either side of the signal line for the ground connections. This is pretty tight and probably overkill. It is important however to tack down the grounds near the unused SNTJs as well as a few bonds on the opposite side, bonding the chip ground to the box ground. It might not seem like it’s necessary, I know, but you can convince your self by measuring the return loss at room temperature on a network analyzer. The seemingly superfluous bonds can make a difference in eliminating spurious resonances up above 7 GHz. In any case, you should evaluate your bonding job on the network analyzer before placing in the fridge so that you don’t have any surprises when you’re cold. If you have done everything well, you should be near -20dB return loss up to 10 GHz. If you have resonances in S11 you will end up limiting the noise power that can leave the junction resulting in noise temperature that may look excessively high at those frequencies, since the shot noise measurement can only infer the system noise temperature reckoned for the entire chain between the junction and your noise measurement instrumentation.

<h3>Box design </h3>



The box design itself is certainly overkill and is the result of some OCD on my part. It is intended to hold the magnet (which really drives the aluminum normal) and also provide a light-tight seal at the top so that the device itself is really seeing cold metal and not anything else. It is made from OFHC. It used to be gold-plated to make people feel good, but I think this is largely unnecessary. However, you will probably need to hit it with ScotchBrite a little before you bolt it on to your cold plate. It is connectorized with a single Southwest Microwave field-replaceable male SMA flanged connector(P/N213-500SF) with a 12 mil pin + teflon sleeve to feed through the 0.125” box wall. You can use any other flanged connector you like as long as it takes the 12 mil pin standard. The solid model and drawings are available to anybody that wants them. Note: we have started putting male connectors on these boxes so you can mount them directly on the RF+DC end of an Anritsu K250 bias tee without an adapter (we also have heard anecdotally that the Anritsu K251 (the one with the gold-colored body)breaks when cold!).


<h3>Embedded magnet</h3> 



    The box is as big as it is so that it can accommodate a beefy cylindrical rare earth magnet (0.375”×0.2”). Depending on what we had available this will either be one magnet or two thinner magnets stacked. As delivered, this magnet will have a little bit of Apiezon N grease gobbed into the well which will, presumably, keep it from rattling around when mounted on your fridge. If you are skeeved out by having giant magnets near your sample(are your isolators and circulators actually well-shielded, too?) then you can take out the magnet. When you’re cold you’ll have to bias well above the gap to get the noise temperature and gain and you will give up the possibility of extracting temperature. Also note that the Southwest SMA connector that we’ve mounted is slightly magnetic itself, so if you’re worried about stuff like that then you’ll have to replace it with a nonmagnetic version (like the ones from EZ-Form). If you are really paranoid you can replace the 0-80 screws holding the board down, too, but in many of the boxes these are now titanium so they’re already ok. NOTE: If you decide to not use the magnet and have to bias well above the gap, keep in mind that the resulting noise output can be enough to drive some amplifiers into bad nonlinear regimes. Specifically, Josephson parametric amplifiers with critical currents of less than 20 µA are known to saturate just above the point that the shot noise overtakes the quantum noise on the curve. If none of this makes sense, but you think you might have a related problem, send me(jose.aumentado@nist.gov)a note!

<h3>Noise power formula</h3>



The formula for the noise power emitted by the shot noise tunnel junction (SNTJ) in the quantum regime (kBT < hf) is: 
$$
N= Gk_BBT_N + \left(\frac{eV + hf}{2k_BT}\right)
\coth{\left(\frac{eV + hf}{2k_BT}\right)} + \left(\frac{eV - hf}{2k_BT}\right) \\ 
\coth{\left(\frac{eV - hf}{2k_BT}\right)}
$$

G is the power gain of the entire system, kB is the Boltzmann constant, B is the bandwidth, TN is  the noise temperature, V is the junction voltage, h is Planck’s constant and f is the frequency at which the noise is measured. This is the full “double-coth” functi on which yields the quantum noise flat-bottom feature. Although the double-coth is overkill for higher temperatures, you can still use whatever fit routine you cookedup for the double-coth fit, it just might take a little longer. In the end,however, if you are just concerned with noise temperature, then the quantum noise stuff and even the temperature don’t matter. You can fix the temperature to something close and just do linear fits on the asymptotes to extract the gain and the noise temperature.


<h3>Operation</h3>



Determine the junction resistance/Voltage bias ratio I find that the resistance of the junctions can increase a few Ωs over their room temperature values when cooled to dilution temperatures. As such, if you’re really gungho about having a 50Ω junction, you’ll do best to start with something like 48 or 49Ω. Since the goal for these
devices is an accurate assessment of the noise temperature, getting closeto 50 Ω is important, but how close is close? If your source impedance (junction impedance) is 41 Ω, then the fraction of noise power that can propagate out is 
$$
\eta= 1−|\Gamma|^2 = 0.99 
$$
so you’ll only end up with a 1% overestimation in your final system noise temperature. With the current design, your main worry will be the loss and reflections introduced by all of the intervening connectors and components. As a rule of thumb we can assume about 0.25 dB for every stage of circulation, so each double circulator will introduce 0.5 dB. All of the couplers and cables will nickel and dime you such that it is very typical toget 2 or 3 dB attenuation by the time your signal reaches your cryogenic HEMT. The bottom line is that you should worry about your wiring and component losses before getting finicky about the junction impedances. That being said, we have done our best to make our shot noise junctions very close to 50Ω anyways. In any case, our conventional wiring is intended to measure this resistance as well as is reasonable, mostly to determine the actual bias voltage that is placed across the junction. For this we like to employ a 3-probe configuration (see Figure 0.1) where we bias through a cold resistor and measure the voltage on the junction through a voltage tap. The actual resistance is unimportant in the sense given above, but the junction voltage is important for scaling voltage to temperature units in the shot noise fit. In fact, if you’re confident that your junction resistance is close enough, you can ignore it and focus on the divider ratio VJ/VB so that you know how to scale your room temperature voltage bias to the voltage that’s actually on the junction. 




This is the number that you’ll use for your fit. It’s important to get this right since the noise temperatures that you will infer will scale with your measurement of the junction voltage through $k_B$.
Figure 0.1: Bias resistor configuration for determining the ratio between the applied bias VB and the junction bias VJ. If you use a current source in place of VB you can determine the resistance and convince yourself that your reflection loss will probably be good enough.


<h3>How do you set the bias resistance values? </h3>



If you’re after noise temperature and your SNTJ is cold enough to be well into the quantum regime, then you’ll be happy if you have something like 10×hf/e, that is, something that well exceeds the corners on the quantum noise flat bottom. If you’re at 6GHz, then this turns out to be something like 25µV. Since I like to have my bias sources scaled so that they swing about ±1V for the range of interest, this means a divider ratio of something like 1:40,000. If you don’t use any room temperature dividers, you can use a single bias resistor value of 2 MΩ. I usually end up using a divider at room temperature and using something like 100kΩ so that I can measure temperature at higher temperatures as a sanity check.


\section{
Measure the noise }



Although you can,in principle, demodulate the output of your amp chain using a diode rectifier, you’ll have to play games setting the filtering of the room temperature mixer output and you’ll have to think about one more thing. I think that it’s far easier to demodulate the output of the SNTJ at a given frequency by using a standard spectrum analyzer in a triggered, zero span acquisition at the frequency that you care to measure. I use the Agilent E4407B for this purpose. Make sure to set it to “sample” each bin and not pull the max/min at each point. The E4407B also lets you set the display to linear Watts. Open up the resolution bandwidth as wide as you need to, usually less than 10 MHz on a typical instrument. This will yield the noise curve integrated over this bandwidth, so if you’re interested in getting the noise over a narrower bandwidth then you can crank it down, but be prepared to wait longer. Anyhow, with the trace averaging in the instrument you can accumulate averaged shot noise mustache traces quickly and easily by syncing a triangle sweep on the bias. Output the data however you like and perform your fit using the scaling ratio VJ/VB that you found with your 3 probe measurement above. Since these automated spectrum analyzers do a lot under the hood, you have to make sure that however it configures its own internal attenuators to level its incoming signals, that its noise floor is dominated by your amplification chain. Set up your bias sweep so that you can see a good noise mustache, sweeping out something like 10 $k_BT$  and setting the power scale on your analyzer so that the max noise fills out the y-scale. If you shut off your cryogenic HEMT, then the noise power shown on the analyzer should be very closeto zero. This is an indication that the cryogenic HEMT is dominating your system noise. If it doesn’t reach zero, then you have other stuff later in the chain that is contributing significant noise and you have some troubleshooting to do.


\section{
HEY! My system noise looks higher than the spec sheet!} 




Your first inclination might be to think that somehow the SNTJ is giving you a wrong answer, probably because the noise looks high and you really want to go out and say that you have quantum-limited added noise. Before you come to this conclusion, you have to remember what the SNTJ gives you—it can only yield the total system noise between the junction and the spectrum analyzer. This includes loss contributions inside the SNTJ box between the junction and the SMA, all of the reflections at every connection, the loss in every circulator, bias tee, and normal metal cable as well as the temperature of those cables. If you are unsure about what these pieces add, or skeptical that they should add so much, then sit down and calculate and then show someone else as a check. We are happy to help you figure out what’s going on. As a point of reference,though,be aware that many of the system noise numbers that many groups have stated in their papers are just plain wrong.  It is typical for instance, fora CalTech HEMT amplifier to have 5K noise temperature, but after all of the loss between a DUT at the mixing chamber and the input of the HEMT at 4 K, we find it typical to see between 10-15 K total system noise.

<h3>Attribution</h3> 



Some groups have asked how to attribute credit to us (co-authorship, acknowledgement, or nothing at all?). It really depends on what piece the SNTJ plays in your experiment, but in the end we should just have a conversation to make sure we’re all on the same page.


<h3>More stuff?</h3> 




These are just random notes off the top of my head. If you have other tips and suggestions that you think should be included, then feel free to write something up and send it to me at jose.aumentado@boulder.nist.gov and I’ll work on including it. Also,if you would like example Matlab code+data,just send us a note.


\end{document}
