
\documentclass[11pt]{article}
\usepackage{graphicx}
\begin{document}

\section{
Social Media in Science}



    Social media has fundamentally altered the Human Condition for all people on the Planet.  Communication in politics, war, the arts, in social interactions and in business all now take place at the pace of Twitter.  And yet science continues to publish on a 18 month time scale.  We can claim this is because we are more careful and subject our work to more review and editing than others, but this is a red herring.  




    On a daily basis we are generating information which is neither proprietary nor secret, is too small and not new enough to warrant publication and yet which can benefit the overall progress of our field enormously.  The lack of real time technical communication throughout the community in a quantitative and math-friendly format which can be interacted with on smart phones is therefore holding back all scientific progress in comparison to other human endeavors today.  




    The shot noise amplifier metrology being developed at APL can benefit enormously from the use of the Web to publish information in real time.  The goal of this project is not to create one single result which will be shared in a paper, or a single technology which is to be physically distributed.  Rather, a successful system will involve a continuous flow of information and materials throughout the community of quantum information research labs.  In terms of physical objects, this flow will consist of devices shipped from the Aumentado lab at NIST, packages and boards from various outsourced suppliers, supplies of quantum amplifiers from research labs and startup companies, commercial HEMT amps, and various cryogenic microwave components.  Right now success in this field is often dependent on so-called "lore": word of mouth information about what components will either fail cold or have small problems which affect overall performance. 




    Given the tight-knit nature of the quantum information and measurement community it should be possible to replace this "lore" of rumors told in emails and verbally with real time quantitative information shared on the Web between labs.  What we propose to do here is to create the infrastructure to do just that: to share information in real time at all phases of the workflow from purchasing components to fabrication, to exchanging of samples between labs, testing with the shot noise junctions, further metrology using qubit readouts, and use of this information for further improveing the measurement.  It is a goal of this work for no lab to create an improvement in measurement noise floor alone, but to always be in contact with all other researchers in a constant exchange of information between the people actually gathering that information as it happens: grad students, post docs, technicians etc.  When a grad student is stuck on a 6 month snag with a mysteriously broken measurement they should have every single other grad student in the field potentially as backup, helping solve their problems in real time.  This can provide exponential speedup to progress, as the current waste of benchtop researcher time trying to fix broken experiments that have the same failure mode another researcher already found but didn't publish is potentially huge.  




    We note that this new class of information channel does not obviate the need for existing systems of publication: peer reviewed papers, white papers, PhD dissertations, etc.  In fact, it has the potential to create a vastly more efficient and effective peer review system, in which we are all able to see each other's work and not just evaluate it as we currently do but actually make it better, so that no one ends up "stranded" in a obscure and broken part of the field where nothing works and no one trusts their results.  



\section{
Summary of Proposed System}




    It is our intent to create a system for rapid mobile-based sharing on the Web which tracks the tunnel junction based shot noise metrology experiment all the way through the workflow of that type of measurement.  The stages of this workflow that we must deal with are:
    

    <ol>
        <li>Junction design</li>
        <li>Junction Fabrication</li>
        <li>Junction distribution</li>
        <li>Package fabrication</li>
        <li>package distribution</li>
        <li>board fabrication</li>
        <li>board distribution</li>
        <li>Installation in fridge</li>
        <li>Overall measurement chain assembly and documentation</li>
        <li>Tests before and during cooldown</li>
        <li>Data at base temperature</li>
        <li>Plotting data</li>
        <li>Interpreting data for future improvements</li>
    </ol>
    
    


        At each phase of this workflow the proposed system will have a method for rapid publication directly to the Web for sharing with peers.  The Geometron geometric metalanguage, developed independently by Lafelabs LLC, allows for rapid creation of graphics using a specific graphical language(an instance of the metalanguage).   
    




\end{document}
